\documentclass[9pt,twoside]{pnas-new}
\usepackage[flushleft]{threeparttable}
\usepackage{tabularx}

\templatetype{pnassupportinginfo}

\title{Predicting High-Risk Opioid Prescriptions Before they are Given}
\author{Justine S. Hastings, Mark Howison, Sarah E. Inman}
\correspondingauthor{Justine S. Hastings.\\E-mail: connect@ripl.org}

\begin{document}

\maketitle

\SItext

\section{Experimental Design}

Our objective was to define a panel of Rhode Island Medicaid recipients who received an initial opioid prescription under Medicaid coverage; define adverse outcomes of opioid dependence, abuse, or poisoning; and model and assess the accuracy of predictions of adverse outcomes using information known only prior to the initial prescription. Data were split into randomly-sampled training, validation, and testing sets using the ratio 50:25:25 at the beginning of the study. We report the results of model predictions on the testing set, which was withheld from analysis prior to the preparation of the manuscript.

Data are from the period 2005-2017, and include Rhode Island administrative records from the Department of Human Services (DHS), Department of Labor and Training (DLT), Department of Corrections (DOC), Medicaid program (under the Executive Office of Health and Human Services), and police agencies (including the Rhode Island State Police and eight municipal police departments).

Although our data span the years 2005 to 2017, we construct a panel of individuals with initial prescriptions between 2006 and 2012 to allow for the construction of variables a year before the initial prescription and to define outcomes up to five years after the initial prescription.

\subsection{Data Availability}

Data are available through individual data sharing agreements with each of the following Rhode Island agencies and municipal police departments: RI Department of Corrections, RI Department of Labor and Training, RI Executive Office of Health and Human Services, RI State Police, Central Falls Police Department, Cranston Police Department, Cumberland Police Department, Middletown Police Department, Narragansett Police Department, Providence Police Department, Warwick Police Department, Woonsocket Police Department.

\section{Panel and Outcome Definitions}

\subsection{Opioid Prescriptions}

To define our panel, we first establish which pharmacy claims correspond to opioid prescriptions. The primary identifier for the dispensed drug is a standardized 11-digit National Drug Code (NDC) from the U.S. Food and Drug Administration's NDC Directory \cite{ndc}. This directory is only available as a current snapshot, and because our claims data start in 2005, there are many unmapped NDCs to the current directory. Out of approximately 14.8 million pharmacy claims between 2006 and 2012, only 66.8 percent join to the current directory. Therefore, we construct a historical NDC directory using a data mining framework that downloads and collates all available Internet Archive snapshots of the FDA's NDC website since 2000 \cite{historicalndc}. This historical directory also includes full ingredient lists for each NDC, standardized to milligrams. Using this improved directory, 88.1 percent of pharmacy claims between 2006 and 2012 map to an NDC entry. 

We define an opioid prescription as any claim for a drug containing an opioid ingredient at or above the recommended starting dose when initiating opioid therapy for chronic pain management, as established in Washington State's 2015 prescribing guideline and further cited in the Centers for Disease Control's 2016 prescribing guideline \cite{amdg, dowell}. Table S9 lists these ingredients and the minimum amounts we use to define an opioid drug. Of the 4,359 drugs containing one of these ingredients, 4,175 meet the minimum threshold amount and appear in 3.9 percent of claims.

Additionally, we define a recovery prescription as any NDC containing one of four ingredients commonly used in medication-assisted treatment of an opioid use disorder, which identifies 412 such drugs that appear in 0.5 percent of claims. These prescriptions may indicate that an individual has a pre-existing opioid use disorder.

\subsection{Opioid Injections}

In addition to identifying opioid prescriptions in pharmacy claims, we also identified procedure codes for epidural or intraveneous opioid injections in all of the Medicaid claims. Table S10 lists these codes, which were identified by searching the descriptions of all procedure codes for each of the opioid ingredients in the Washington State prescribing guidelines used to identify prescription drugs.

\subsection{Outcomes}

For each individual in our panel, we examine all of the Medicaid claims following their initial opioid prescription to construct indicator variables for four types of adverse outcomes: opioid dependence, opioid abuse, prescription-opioid poisoning and heroin poisoning. We include heroin poisoning as an outcome given the increasing use of heroin among those who abuse opioids, and the high proportion (greater than 80 percent) of joint heroin-prescription-opioid users who abused opioids prior to using heroin \cite{jones}.

We determine these outcomes from the claim's International Classification of Diseases (ICD) diagnosis codes, which are used by medical professionals to classify a patient's health conditions following an encounter. Because our data span the transition from the ICD-9 to ICD-10 classification, we include diagnosis codes from both. Table S11 lists the codes used to indicate each of these four diagnosis-related outcomes. 

Not everyone with an opioid use disorder receives a diagnosis code. Though it is unknown precisely what fraction of opioid use disorders go undiagnosed, Carrell {\em et al}. found that diagnosis codes were missing for as many as a quarter of patients for whom their providers were aware of opioid abuse \cite{carrell}. Similarly, a study by Barocas {\em et al}. estimated that only 44\% of individuals with opioid use disorder were identified as such in claims and administrative records \cite{barocas}. To address the challenges with diagnoses codes, we define a fifth treatment outcome using procedure codes related to the treatment of opioid use disorder, and more generally for drug rehabilitation and detoxification (see ``Treatment'' in Table S11). Finally, we define a sixth ``any'' outcome as the union of any of the diagnoses or treatment outcomes, to capture as broad a population of individual with opioid use disorder as possible. Data and measurement limitations notwithstanding, our model demonstrates that administrative data can be combined to form an accurate prediction of these outcomes, suggesting a feasible path forward for utilizing data to inform prescription risk. Figure S1 shows the accumulating fraction of adverse outcomes over the five year period following initial prescription.

\subsection{Final Panel}

Out of 400,024 distinct Medicaid enrollees between 2006 and 2012, our panel initially contains 85,377 individuals who received at least one opioid prescription or injection in that period. We exclude 500 individuals who received a recovery prescription before their initial opioid prescription or injection, since this indicates they may have been seeking treatment for an opioid use disorder. We exclude 4,109 individuals with an adverse outcome prior to their initial opioid prescription or injection, since we assume they were already receiving opioids from another source, such as through private insurance before enrolling in Medicaid. Our final panel includes 80,768 individuals. Table S1 shows the incidence of adverse outcomes among these individuals by baseline characteristics.

\section{Variable Construction}

We construct variables that summarize information known in the 12 months prior to the individual's initial prescription.

Using the demographics from the integrated RI 360 database \cite{hastings}, we construct variables for (modal) age, sex, race, marital status, body mass index, and median income and fraction below the federal povery line in the home Census block group. Using DHS data, we construct variables for household size and new births in the household, and monthly payments for the Supplemental Nutrition Assistance Program (SNAP), the Temporary Assistance for Needy Families (TANF), the General Public Assistance (GPA), the Child Care Assistance Program (CCAP), and State Supplemental Payment portions of Supplemental Security Income benefits. Using DLT data, we construct indicators for sector of work derived from the first two digits of industry codes assigned according to the North American Industry Classification System (NAICS); monthly payments for Temporary Disability Insurance (TDI) and Unemployment Insurance (UI); and quarterly wage history, including average quarterly wages and variance, the number of employers and the number of hours worked (for hourly employees); the monthly unemployment rate in Rhode Island; and the annual national unemployment rate for two-digit NAICS industries that the individual has worked in. Using DOC data, we construct indicators for charges, seven categories of sentencing, and commitments and releases from prison. Using police data, we construct variables for arrests; the number of car crashes involved and injured in; and the number of and total fines for citations.

The largest set of variables comes from the Medicaid data. These include indicators for enrollment eligibility categories, plan type, and payer codes; number of claims and total bill and payment amounts for all claims and for Emergency Department claims; indicators for prescriptions in 262 drug categories from the AHFS Pharmacologic/Therapeutic Classification;\footnote{{\em AHFS\textsuperscript \textregistered Pharmacologic/Therapeutic Classification\textsuperscript \textcopyright} used with permission. \textcopyright\ 2019, the American Society of Health-System Pharmacists, Inc. (ASHP). The Data is a part of the {\em AHFS Drug Information\textsuperscript \textregistered}; ASHP is not responsible for the accuracy of transpositions from the original context.} and topic models summarizing the concatenated text descriptions for all of the individual's ICD-9 diagnosis codes and HCPCS procedure codes. We also include summary counts of the number of distinct diseases using the Clinical Classifications Software \cite{ccs}, of distinct chronic conditions using the Chronic Condition Indicators \cite{cci}, and of distinct procedure codes.

\subsection{Topic Modeling}

We construct the topic models using a technique called non-negative matrix factorization (NMF), which is commonly used in text analysis to discover latent topic structure in documents \cite{fevotte}. In this application, we treat each individual's concatenated text descriptions of diagnosis and procedure codes as a document to learn the latent topic structure across individuals' health histories. Our topic models summarize 80,768 documents comprised of 16,367 distinct words from the code descriptions, after removing 173 uninformative words using a stopword list. The total corpus consists of over 20.5 million words.

NMF works by factorizing the non-negative $d \times w$ matrix of the documents' word frequencies into non-negative matrices $d \times t$ and $t \times w$, where $d$ is the number of documents, $w$ is the number of distinct words, and $t$ is the number of topics. We apply a term frequency-inverse document frequency (TF-IDF) transformation to the $d \times w$ matrix to reweight the word frequencies by their overall frequencies in the entire corpus, which is common practice when implementing NMF. The $d \times t$ matrix represents the weighting of topics for each document, and the $t \times w$ matrix represents the weighting of words for each topic. We summarize each topic using the 10 words with the greatest frequency in the $t \times w$ matrix.

Because the number of topics $t$ is not known \emph{a priori}, we tune this parameter by finding the $t$ with the best out-of-sample area under the operating-receiver characteristic curve (AUC) in a logistic regression that includes only the topic model variables. We use only the training set for this tuning, and further subdivide it in half into topic training and topic validation sets. We consider an increasing number of topics and terminate the tuning procedure when the AUC does not improve by more than 0.001. The tuning achieves AUCs on the topic validation set of 0.663 for 10 topics, 0.670 for 20 topics, 0.684 for 50 topics, 0.685 for 100 topics. Therefore, we select the model with 50 topics for the final variables.

\subsection{Low-Dosage Opioids}

Within the prescription drug categories, there is a category for opiate agonists. By construction of our panel, no individuals should have previously received an opioid prescription. However, the opiate agonist category includes 152 drugs that were not identified in the 4,175 opioid drugs from our historical NDC directory, and which are listed in Table S3. These drugs either contain an opioid ingredient at a lower amount than the minimum thresholds defined by the Washington State prescribing guidelines, or contain an ingredient not identified in those guidelines (e.g., ``opium''). Therefore, the opiate agonist variable indicates that the individual received a drug that was not likely for initiating opioid therapy, but nonetheless contains a small amount of an opioid ingredient. Most of these drugs are over-the-counter cough syrups or painkillers combined with small amounts of an opioid ingredient. Of the 152, there are eight that are not present in the historical NDC directory, possibly because they were on the market for a short enough time that they do not occur in any of the available historical snapshots of the NDC directory.

\subsection{Tensors}

For our neural network models, we construct tensors of monthly values for a given variable for each of the individuals in our panel in the 12 months prior to the individual's initial prescription. Missing values are imputed using mean values from the training population.

The DHS tensor includes 13 variables for demographics (age and indicators for sex, race, and Spanish or Portuguese as a primary language) and monthly payments for the Supplemental Nutrition Assistance Program (SNAP), the Temporary Assistance for Needy Families (TANF), the General Public Assistance (GPA), the Child Care Assistance Program (CCAP), and State Supplemental Payment portions of Supplemental Security Income benefits.

The DLT tensor includes 31 variables for indicators for sector of work derived from the first two digits of industry codes assigned according to the North American Industry Classification System (NAICS); monthly payments for Temporary Disability Insurance (TDI) and Unemployment Insurance (UI); and quarterly wage history, including wage amount, the number of employers and the number of hours worked (for hourly employees).

The DOC tensor includes 16 variables for demographics (age and indicators for sex, race, Spanish as a primary language), and indicators for charges, seven categories of sentencing, and commitments and releases from prison.

The Medicaid tensor includes 683 variables for demographics (age and indicators for sex, race, and Spanish or Portuguese as a primary language); indicators for eligibility categories, plan type, and payer codes at each month of enrollment; number of claims and total bill and payment amounts for all claims and for Emergency Department claims; the number of prescriptions in each of 265 categories from the AHFS Pharmacologic/Therapeutic Classification; and indicators for ICD-9 diagnosis codes and HCPCS procedure codes for all codes that are correlated >0.02 with any adverse outcome in the training population.

The police tensor includes 42 variables for demographics (age and indicators for sex and officer-observed race); indicators for all arrests, DUI arrests, and domestic-offense arrests; the number of car crashes involved and injured in; the number of and total fines for citations; and the spatio-temporal intensity of calls for service in the individual's home Census block group for 29 categories of calls.

Finally, we construct an integrated tensor including all of the 785 variables from the DHS, DLT, DOC, Medicaid, and police tensors. The dimension of this integrated tensor are 70,153 individuals x 12 months x 785 variables.

\section{Models}

We estimate a range of predictive models using modern machine learning algorithms, which vary in both their complexity and interpretability. For example, a class of models called ``regularized regression models'' estimate standard linear models, but search over many potential explanatory variables, potentially more explanatory variables than available data observations, to maximize out-of-sample predictive fit and minimize overfitting. Like ordinary least squares or logistic models, the model results are easy to interpret, but the complexity is limited to functions of variables the researcher specifies in advance. At the other extreme are artificial neural network models where the algorithm searches over non-linear transformations of layers of local linear regressions. The increased complexity allows the algorithm to search for arbitrary non-linearities and interactions between variables, but at a cost of greatly reducing the interpretability of the model (e.g., it is difficult to simply measure which variables contribute most to predictive fit).

\subsection{Regularized Regression}

For our regularized regression, we use an algorithm called Bootstrap Least Absolute Shrinkage and Selection Operator (BOLASSO) \cite{bach}. This algorithm is a generalization of the popular LASSO algorithm which is able to consistently identify a model even when predictors are highly correlated. The BOLASSO selects the predictors with non-zero coefficients that appear in at least 90\% of bootstrapped LASSO models. 

Following convention, we use BOLASSO to select the variables from among 560 variables which are persistently the strongest predictors of future adverse opioid outcomes, and we present results from a second-stage logistic regression of an indicator for future adverse outcomes on these selected variables, to describe the predictive power of each variabe. Exhibit A6 lists the variables selected by the BOLASSO as occuring with a non-zero coefficient in more than 90 of the 100 LASSO bootstrap replicates, along with the regression results from the second-stage logistic regression. In addition to the second-stage logistic regression, we also construct a regression ensemble model that averages the predictions of all 100 bootstrap replicates in the BOLASSO.

We fit each LASSO bootstrap replicate on the training set using a regularized logistic regression implementation called the gamma LASSO, which was developed specifically to address the challenges of modeling sparse, high-dimensional data \cite{taddy}. Since a predictive model fits idiosyncratic noise through increased complexity in the model's structure, machine learning techniques commonly penalize complexity in the models they produce through a process called regularization. We tune the regularization parameters for the gamma LASSO model through a parameter search over gamma values in [0, 1, 10] and a path of 100 lambda values, and we select the model with the best area under the receiver-operating characteristic curve (AUC) on the validation set. Regularization helps prevent overfitting to the training data and thus improves out-of-sample fit. We are primarily interested in out-of-sample performance since our goal is to use the model to inform successful policy interventions, which require making predictions on new observations \cite{kleinberg}.

\subsection{Neural Networks}

We train a neural network model for each tensor using the Python package Keras \cite{chollet}, which provides an interface to the TensorFlow library \cite{abadi}. Specifically, we train a recurrent neural network (RNN), since RNNs have the ability to model temporal patterns in the input data. We input our training data into a two-layer network of 12x12 Long Short-Term Memory (LSTM) \cite{hochreiter} units with the $\tanh$ activation function. We input the last LSTM layer into a dense layer that applies a sigmoid activation function to the weighted sum of the 10 inputs in order to produce a single predicted probability of adverse outcome. We employ regularization prior to each layer in the form of a dropout factor of 0.25, which causes a random deactivation of units within the layer during training with a fixed probability of 0.25 \cite{srivastava}.

The neural networks are optimized to minimize the binary cross-entropy, also known as log-loss, on the training data. We use the Adam optimization algorithm \cite{kingma}, training with a batch size of 16. We tune the model on the validation set by allowing the neural network to train for as many epochs as needed until the area under the receiver-operating curve (AUC) from predictions on the validation set does not improve by 0.001. Table S12 shows the AUC from predictions on the testing set for each data source and each individual outcome.

\section{Population Estimates}

To estimate population-level characteristics of Medicaid enrollees in Rhode Island, we constructed a second panel of longterm Medicaid enrollees. We included all enrollees who were enrolled for at least six out of 12 months in each of the five years between 2007 and 2011. This panel comprises 120,584 enrollees, who were enrolled with a median of 60 months (interquartile range of 59 to 60 months). Using this panel, we estimated the fraction of adverse outcomes, race/ethnicity, and median age among all enrollees and only those who received an opioid prescription, an opioid injection, both, or neither (see Table S5). For those with an opioid prescription, we calculated the average number of visits in the 30 days prior to the prescription, and the average distance to the five closest providers based on the Census block group of the last known home address before the prescription (see Table S7).

\section{Estimation of Adverse Outcome Cost}

In 2015, 33,091 people died from drug overdoses involving opioids \cite{nida1}, and 2,375,000 individuals over the age of 12 had an opioid use disorder \cite{cbhsq}. The U.S. Department of Transportation's Value of a Statistical Life is \$10.1 million. Florence \textit{et al}. \cite{florence} estimate the aggregate annual societal cost of an opioid use disorder to be \$61,297 (including additional cost of health care, substance abuse treatment, lost productivity, and criminal justice activities). Weiss and Rao \cite{weiss} estimate a 50 percent recovery probability after one year of medication-assisted treatment. Using these statistics, with the simplifying assumption that once an individual receives a prescription, they either overdose resulting in death, become dependent but successfully recover after one year of treatment, or continue to be dependent for ten years, we estimate a ballpark present discounted value of \$450,000 for $C_A$ (see Table S12).

\footnotesize
\bibliography{references}
\normalsize

\begin{table}
\caption{Descriptive statistics for the final panel.}
\centering
\input{tables/TableS1}
\end{table}

\begin{table}
\caption{Regression output for the post-BOLASSO regression (a logistic regression of variables selected by BOLASSO as occurring with a non-zero coefficient in more than 90\% of LASSO bootstrap replicates).}
\centering
\small
\input{tables/TableS2}
\hrule
\begin{tablenotes}
\footnotesize
\item * The frequency can be less than 90\% for variables that were included in the post-BOLASSO as base terms of a selected interaction term.
\end{tablenotes}
\end{table}

\begin{table}
\caption{Low-dosage prescription opioids identified by the AHFS Pharmacologic/Therapeutic Classification category for opiate agonists.}
\centering
\footnotesize
\input{tables/TableS3}
\hrule
\begin{tablenotes}
\footnotesize
\item * NDC code exists in AHFS Pharmacologic/Therapeutic Classification but does not exist in NDC directory.
\end{tablenotes}
\end{table}

\begin{table}
\caption{Estimation of the adverse outcome cost.}
\centering
\begin{tabular}{lr}
Cost of poisoning$^1$ & \$140,735 \\
Cost of successful treatment after 1 year$^2$ & \$30,649 \\
Cost of relapsed treatment for 10 years$^3$ & \$269,282 \\
\hline
\em Total adverse outcome cost: & \em \$440,666 \\[1em]
\end{tabular}
\hrule
\begin{tablenotes}
\footnotesize
\item 1. The cost of poisoning is estimated as the product of the probability of poisoning (0.014) and the DOT Value of a Statistical Life (\$9,600,000 in 2015 dollars \cite{vsl}; \$10,100,770 in inflation-adjusted 2018 dollars). The probability of poisoning is estimated as the number of deaths from drug overdoses related to opioids in 2015 (33,091 \cite{nida1}) divided by the number of persons aged 12 or older estimated to misuse opioids in 2015 (2,375,000 \cite{nida2}).
\item 2. The cost of successful treatment is estimated as the product of the probability of successful remission following treatment for an opioid use disorder (0.5 \cite{weiss}) and the estimated annual societial cost of a non-fatal opioid use disorder (\$56,990 in 2013 dollars \cite{florence}; \$61,297 in inflation-adjusted 2018 dollars).
\item 3. The cost of relapsed treatment is estimated as the product of the probability of relapse following treatment (0.5) and the present discounted value of 10 years of the annual societal cost of a non-fatal opioid use disorder (\$538,564).
\end{tablenotes}
\end{table}

\begin{table}
\caption{Population estimates based on a five-year panel of longterm Medicaid enrollees.}
\centering
\begin{tabular}{lrrrrr}
 & \em All & \em Opioid Rx & \em Opioid Injection & \em Both & \em Neither \\[0.5em]
\em N & 120,584 & 29,623 & 11,916 & 4,842 & 83,887 \\
Adverse Outcome & 4,545 (3.8\%) & 983 (3.3\%) & 732 (6.1\%) & 218 (4.5\%) & 1,114 (1.3\%) \\
White & 59,756 (49.6\%) & 17,465 (59.0\%) & 7,814 (65.6\%) & 3,210 (66.3\%) & 37,687 (44.9\%) \\
African-American & 13,480 (11.2\%) & 3,419 (11.5\%) & 1,006 (8.4\%) & 498 (10.3\%) & 9,553 (11.4\%) \\
Hispanic & 22,305 (18.5\%) & 3,112 (10.5\%) & 1,001 (8.4\%) & 418 (8.6\%) & 18,610 (22.2\%) \\
Median Age & 24 & 34 & 50 & 42 & 12
\end{tabular}
\end{table}

\begin{table}
\caption{Calculations of the difference in False Discovery Rate (FDR) between whites and minorities that can be detected given our sample size at a power of 0.8.}
\centering
\begin{tabular}{lrrrrr}
\em Decile & \em N White & \em N Minority & \em FDR White & \em FDR Minority$^*$ & \em FDR Difference$^*$ \\[0.5em]
1 & 1,736 & 218 & 0.762 & 0.844 & 0.082 \\
2 & 3,319 & 521 & 0.815 & 0.864 & 0.049 \\
3 & 4,863 & 843 & 0.850 & 0.886 & 0.036 \\
4 & 6,397 & 1,136 & 0.875 & 0.904 & 0.029 \\
5 & 7,829 & 1,483 & 0.890 & 0.914 & 0.024 \\
6 & 9,098 & 1,910 & 0.902 & 0.922 & 0.020 \\
7 & 10,074 & 2,540 & 0.909 & 0.926 & 0.017 \\
8 & 10,851 & 3,336 & 0.915 & 0.930 & 0.015 \\
9 & 11,391 & 4,344 & 0.918 & 0.931 & 0.013 \\
10 & 11,790 & 5,571 & 0.921 & 0.933 & 0.012 \\[1em]
\end{tabular}
\hrule
\begin{tablenotes}
\footnotesize
\item * The FDR for minorities and the difference in FDR was calculated from the other parameters using the \texttt{power twoprop} command in Stata version 14.2 (StataCorp LLC, College Station, TX).
\end{tablenotes}
\end{table}

\begin{table}
\caption{Population estimates of access to health care providers by minority status and adverse outcome status.}
\centering
\begin{tabular}{lrr}
\em Group & \em Average visits in 30 days & \em Average distance to closest five providers at the last known \\
 & \em prior to initial opioid prescription & \em home address prior to initial opioid prescription \\[0.5em]
White & 1.10 & 1.25km \\
African-American & 1.12 & 0.89km \\
Hispanic & 1.09 & 0.85km \\[0.5em]
Adverse outcome & 1.03 & 1.14km \\
No adverse outcome & 1.04 & 1.15km
\end{tabular}
\end{table}

\begin{table}
\caption{Cost of medication-assisted treatment (MAT) for 1,000 individuals assuming a 50\% remission rate and annual MAT cost of \$6,552 (low) to \$14,112 (high).}
\centering
\begin{tabular}{lrrrr}
\em Year & \em Cost (Low) & \em Cost (High) & \em In Remission & \em \% In Remission \\[0.5em]
1 & \$3,276,000 & \$7,056,000 & 500 & 50\% \\
2 & \$1,638,000 & \$3,528,000 & 750 & 75\% \\
3 & \$819,000 & \$1,764,000 & 875 & 88\% \\[0.5em]
\em Total & \em \$5,733,000 & \em \$12,348,000 & &
\end{tabular}
\end{table}

\begin{table}
\caption{Minimum amounts of ingredients in a drug to classify it as an opioid prescription or a recovery prescription.}
\centering
\begin{tabular}{lc}
\em Opioid Ingredient & \em Minimum Amount (mg) \\[0.5em]
Codeine & 30.0 \\
Fentanyl & 0.0125 \\
Hydrocodone & 5.0 \\
Hydromorphone & 2.0 \\
Meperidine* & 0.0 \\
Morphine & 10.0 \\
Oxycodone & 5.0 \\
Oxymorphone & 5.0 \\
Tapentadol & 50.0 \\
Tramadol & 50.0 \\[1em]
\em Recovery Ingredient° & \em Minimum Amount (mg) \\[0.5em]
Buprenorphine & 0.0 \\
Methadone & 0.0 \\
Naloxone & 0.0 \\
Naltrexone & 0.0 \\[1em]
\end{tabular}
\hrule
\begin{tablenotes}
\footnotesize
\item * Meperidine has no recommended starting dose for treatment of chronic pain because of its risk for complications in older adults; therefore, we consider any amount as evidence that the drug is an opioid.
\item ° We consider any amount of a recovery ingredient as evidence that the drug may have been used to treat a prior opioid use disorder.
\end{tablenotes}
\end{table}

\begin{table}
\caption{Procedure codes used to identify opioid injections.}
\centering
\begin{tabular}{ll}
\em Code & \em Description \\[0.5em]
J2270 & Injection, morphine sulfate, up to 10 mg \\
J2271 & Injection, morphine sulfate, 100mg \\
J2275 & Injection, morphine sulfate (preservative-free sterile solution), per 10 mg \\
Q9974 & Injection, morphine sulfate, preservative-free for epidural or intrathecal use, 10 mg \\
S0093 & Injection, morphine sulfate, 500 mg (loading dose for infusion pump \\
J2274 & Injection, morphine sulfate, preservative-free for epidural or intrathecal use, 10 mg \\
J2410 & Injection, oxymorphone hcl, up to 1 mg \\
J1170 & Injection, hydromorphone, up to 4 mg \\
S0092 & Injection, hydromorphone hydrochloride, 250 mg (loading dose for infusion pump) \\
J0745 & Injection, codeine phosphate, per 30 mg \\
J3010 & Injection, fentanyl citrate, 0.1 mg \\
J1810 & Injection, droperidol and fentanyl citrate, up to 2 ml ampule \\
J2175 & Injection, meperidine hydrochloride, per 100 mg \\
J2180 & Injection, meperidine and promethazine hcl, up to 50 mg
\end{tabular}
\end{table}

\begin{table}
\caption{Diagnosis and procedure codes used to indicate adverse outcomes when occurring in any claim after the initial opioid prescription.}
\centering
\begin{tabular}{lll}
\em Outcome & \em Code & \em Description \\[0.5em]
\textbf{Opioid} & 304.0 & Opioid type dependence \\
\textbf{Dependence} & 304.7 & Combinations of opioid type drug with any other drug dependence \\
 & F11.2* & Opioid dependence \\[0.5em]
\textbf{Opioid Abuse} & 305.0 & Nondependent opioid abuse  \\
 & F11.1* & Opioid abuse \\[0.5em]
\textbf{Prescription-} & 965.00 & Poisoning by opium (alkaloids), unspecified \\
\textbf{Opioid} & 965.02 & Poisoning by methadone \\
\textbf{Poisoning} & 965.09 & Poisoning by other opiates and related narcotics \\
 & 970.1 & Poisoning by opiate antagonists \\
 & E850.1 & Accidental poisoning by methadone \\
 & E850.2 & Accidental poisoning by other opiates and related narcotics \\
 & E935.1 & Methadone causing adverse effects in therapeutic use \\
 & E935.2 & Other opiates and related narcotics causing adverse effects in therapeutic use \\
 & E940.1 & Opiate antagonists causing adverse effects in therapeutic use \\
 & T400* & Poisoning by, adverse effect of and underdosing of opium \\
 & T402* & Poisoning by, adverse effect of and underdosing of other opioids \\
 & T403* & Poisoning by, adverse effect of and underdosing of methadone \\[0.5em]
\textbf{Heroin} & 965.01 & Poisoning by heroin \\
\textbf{Poisoning} & E850.0 & Accidental poisoning by heroin \\
 & E935.0 & Heroin causing adverse effects in therapeutic use \\
 & T401* & Poisoning by and adverse effects of heroin \\[0.5em]
\textbf{Treatment} & J2310° & Naloxone HCI Injection, per 1 mg \\
 & J2315° & Naltrexone injection, depot form, 1mg \\
 & J0592° & Buprenorphine HCL injection, 0.1mg \\
 & X0305° & Methadone detoxification – outpatient \\
 & X0321° & Methadone maintenance, assessment and evaluation, counseling, treatment \\
 &  & and review, and lab testing \\
 & H0020° & Alcohol and or drug services; methadone administration and or service \\
 & J1230° & Injection, methadone, up to 10mg \\
 & 83840° & methadone \\
 & 946° & Alcohol and drug rehabilitation and counseling \\
 & 9464° & drug rehabilitation \\
 & 9465° & drug detoxification \\
 & 9466° & drug rehabilitation and detoxification \\
 & 9467° & combined alcohol and drug rehabilitation \\
 & 9468° & combined alcohol and drug detoxification \\
 & 9469° & combined alcohol and drug rehabilitation and detoxification \\[1em]
\end{tabular}
\hrule
\begin{tablenotes}
\footnotesize
\item * ICD-10 diagnosis code
\item ° HCPCS procedure code
\end{tablenotes}
\end{table}

\begin{table}
\caption{Area under the receiver-operating characteristic curve (AUC) of neural network models using different subsets of administrative data and outcome definitions. Confidence intervals are calculated from 100 bootstrap replicates.}
\centering
\input{tables/TableS12}
\end{table}

\begin{table}
\caption{Predictors of injection status before and after propensity score weighting.}
\centering
\input{tables/TableS13}
\end{table}

\begin{figure}
\caption{Cumulative frequency of adverse outcomes over time since initial opioid prescription. Adverse outcomes are indicated by the diagnosis and procedure codes in Medicaid claims following the initial prescription. An individual may experience multiple types of adverse outcomes, and ``any" is the union of the five specific outcome types. Opioid dependence is the most prevalent of the types.}
\centering
\includegraphics[width=6in]{figures/FigureS1}
\end{figure}

\begin{figure}
\caption{The fraction of true outcomes in the test sample. The vertical black line indicates the base rate of outcomes among the entire population, which is 0.057.}
\centering
\includegraphics[width=6in]{figures/FigureS2}
\end{figure}

\begin{figure}
\caption{The break-even cost ratio for minority status (a), incarceration history (b), and disability status (c). Error bars indicate the 95\% confidence interval calculated from 100 bootstrap replicates.}
\centering
\includegraphics[width=6.8in]{figures/FigureS3}
\end{figure}

\begin{figure}
\caption{Odds ratios from the post-BOLASSO regression using variables from Medicaid data only (c.f. Figure 1). Those $<0.75$ and $>1.25$ are labeled.}
\centering
\includegraphics[width=6.8in]{figures/FigureS4}
\end{figure}

\begin{figure}
\caption{The break-even cost ratio for three values of the effectiveness rate $\alpha$ for Medicaid data only (c.f. Figure 2).}
\centering
\includegraphics[width=3.4in]{figures/FigureS5}
\end{figure}

\begin{figure}
\caption{The false discovery rate for minority status (a), incarceration history (b), and disability status (c) for Medicaid data only (c.f. Figure 3).}
\centering
\includegraphics[width=6.8in]{figures/FigureS6}
\end{figure}

\end{document}
